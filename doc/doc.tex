\documentclass[12pt, a4paper]{article}

\usepackage[margin=2.2cm]{geometry}

\usepackage[utf8]{inputenc}
\usepackage[hidelinks]{hyperref}
\usepackage{amsmath}
\usepackage{amsfonts}
\usepackage{amssymb}
\usepackage{graphicx}
\usepackage{caption}
\usepackage{subcaption}
\usepackage{sidecap}
\usepackage{listings}
\renewcommand\refname{Referências}
\title{Estatística e Probabilidades: TP}
\date{Maio de 2014}

\author{
    André Taiar (2014032917) \emph{andre.taiar@gmail.com}\\
}

\begin{document}
\maketitle

\section{Especificação}

Este trabalho consiste em implementar um programa que calcule a probabilidade de existirem valores
da $Binomial(n,p)$ entre valores $m_0$ e $m_1$ de duas formas diferentes:

\begin{enumerate}
    \item utilizando a definição da Binomial;
    \item utilizando a aproximação da Normal para a distribuição Binomial.
\end{enumerate}

\section{Desenvolvimento}

\subsection{Visão geral}

Ambas as implementações foram feitas no mesmo programa que pode ser encontrado permanentemente no
link \url{https://github.com/taiar/est-prob/blob/master/binomial_2_normal.c}. O programa foi escrito
na linguagem C.

O programa pode ser compilado com o compilador gcc através do comando:

\begin{verbatim}
gcc -o prob binomial_2_normal.c -g -Wall -lm
\end{verbatim}

Ao ser executado sem parâmetros, o programa mostra como deve ser utilizado:

\begin{verbatim}
taiar@guestxor:~/dev/est-prob$ ./prob
Ajuda para executar o programa:

Utilize a linha de comando da seguinte forma:
    ./prob n p m0 m1 algoritmo
 aonde:
    - n: parametro n da distribuicao;
    - p: parametro p da distribuicao (0 < p < 1);
    - m0: menor abcissa da curva de probabilidade;
    - m1: maior abcissa da curva de probabilidade;
    - algoritmo: seleciona o algoritmo da distribuição
      `1` para binomial ou `2` para normal.
\end{verbatim}

\subsection{Código-fonte}

Segue o código fonte das principais passagens do programa comentados:

\begin{lstlisting}[language=c]

...

// Valores das probabilidades da tabela normal
static const double valores[] = {
  0.0000, 0.0040, 0.0080, 0.0120, 0.0160, 0.0199, 0.0239, 0.0279...
};

// Indices das probabilidades da tabela normal
static const char* indices[] = {
  "0.000000", "0.010000", "0.020000", "0.030000", "0.040000"...
};

// Maximo de indices na tabela normal
static int max = 310;

// Busca valores dentro da tabela normal
double phi(double n) {
  char valor[30];
  int i, indice = 0;

  if(n < 0) return 1 - phi((-1) * n);

  sprintf(valor, "%lf", n);
  for (i = 0; i < max; ++i)
    if(strcmp(valor, indices[i]) == 0) {
      indice = i;
      break;
    }
  return valores[indice];
}

...

// Funcao de calculo do fatorial
int fatorial(int n) {
  int ret = 1, i = 0;
  for(i = 1; i <= n; ++i) ret *= i;
  return ret;
}

// Funcao de calculo da selecao
int selecao(int n, int m) {
  if (n < m) return 0;
  return (fatorial(n))/(fatorial(m)*(fatorial(n-m)));
}

// Calculo da binomial
double binomial(double n, double p, double r) {
  return selecao(n, r) * pow(p, r) * pow((1 - p), (n - r));
}

// Calculo da esperanca da Binomial
double binomial_esperanca(double n, double p) {
  return (n * p);
}

// Calculo da variancia da Binomial
double binomial_variancia(double n, double p) {
  return (n * p)*(1 - p);
}

// Funcao que calcula o valor da binomial em um intervalo
double aplica_binomial_intervalo(double n, double p, double m_0, double m_1) {
  int i;
  double acum = 0;
  for(i = m_0; i <= m_1; i += 1)
    acum += binomial(n, p, i);
  return acum;
}

// Mostra a saida do calculo utilizando a binomial
void aplica_binomial(double n, double p, double m_0, double m_1) {
  printf("Distribuicao Binomial");
  printf("Distribuicao:X~B(%lf,%lf)", n, p);
  printf("Intervalo:[%lf,%lf]",m_0,m_1);
  printf("Probabilidade:P(%lf<=X<=%lf)=%lf", m_0, m_1,
    aplica_binomial_intervalo(n,p,m_0,m_1));
}

// Funcao que calcula o valor da normal em um intervalo
double normal_padrao_intervalo(double m_0, double m_1) {
  double p_0 = phi(m_0);
  double p_1 = phi(m_1);
  return (p_1 > p_0) ? (p_1 - p_0) : (p_0 - p_1);
}

// Mostra a saida do calculo utilizando a Normal
void aplica_normal(double n, double p, double m_0, double m_1) {
  double b_esp = binomial_esperanca(n, p);
  double b_var = binomial_variancia(n, p);
  double b_des = sqrt(b_var);
  double n_0   = (m_0 - b_esp)/b_des;
  double n_1   = (m_1 - b_esp)/b_des;
  printf("Distribuicao Normal");
  printf("Binomial:X~B(%lf,%lf)",n,p);
  printf("Normal:Z~N(%lf,%lf)",b_esp,b_var);
  printf("Probabilidade:P(%lf<=Z<=%lf)~P(%lf<=Z<=%lf)=%lf",
    m_0,m_1,n_0,n_1,normal_padrao_intervalo(n_0,n_1));
}

// Execucao do programa
int main(int argc, char **argv) {
  double n, p, m_0, m_1;
  int algoritmo;
  ...
  if(algoritmo == 1)
    aplica_binomial(n, p, m_0, m_1);
  else if(algoritmo == 2)
    aplica_normal(n, p, m_0, m_1);
  ...
}

// Fim do codigo

\end{lstlisting}

\newpage

\section{Execução e testes}

Segue algumas execuções do programa utilizando calculando probabilidades através dos dois algoritmos
implementados:

\subsection{Exemplo 1}

$
n = 255     \\
p = 0.2     \\
m_0 = 39    \\
m_1 = 48
$

\begin{verbatim}
taiar@guestxor:~/dev/est-prob$ ./prob 225 0.2 39 48 1
 ## Distribuicao Binomial ##
Distribuição:   X ~ B(225.000000, 0.200000)
Intervalo:  [39.000000, 48.000000]
Probabilidade:  P(39.000000 <= X <= 48.000000) = 0.585300
taiar@guestxor:~/dev/est-prob$ ./prob 225 0.2 39 48 2
 ## Distribuicao Normal ##
Binomial:   X ~ B(225.000000, 0.200000)
Normal:     Z ~ N(45.000000, 36.000000)
Probabilidade:  P(39.000000 <= Z <= 48.000000)
 ~ P(-1.000000 <= Z <= 0.500000) = 0.532800
\end{verbatim}

\subsection{Exemplo 2}

$
n = 3     \\
p = 0.5     \\
m_0 = 1    \\
m_1 = 3
$

\begin{verbatim}
taiar@guestxor:~/dev/est-prob$ ./prob 3 0.5 1 3 1
 ## Distribuicao Binomial ##
Distribuição:   X ~ B(3.000000, 0.500000)
Intervalo:  [1.000000, 3.000000]
Probabilidade:  P( 1.000000 <= X <= 3.000000) = 0.875000
taiar@guestxor:~/dev/est-prob$ ./prob 3 0.5 1 3 2
 ## Distribuicao Normal ##
Binomial:   X ~ B(3.000000, 0.500000)
Normal:     Z ~ N(1.500000, 0.750000)
Probabilidade:  P(1.000000 <= X <= 3.000000)
 ~ P(-0.577350 <= Z <= 1.732051) = 0.6781
\end{verbatim}

% \begin{thebibliography}{99}

% \bibitem{citeMNist}
% 	Fonte da base de dados MNist, com uma relação da performance de diferentes classificadores.\\
% 	\url{http://yann.lecun.com/exdb/mnist/}

% \bibitem{citeCNN}
% 	Y. LeCun, L. Bottou, Y. Bengio, and P. Haffner, ``Gradient-Based Learning Applied to Document Recognition'' Proceedings of the IEEE, vol. 86, no. 11, pp. 2278-2324, Nov. 1998.
% \end{thebibliography}

\end{document}

